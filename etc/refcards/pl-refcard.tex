%&mex
%=====================================================================
% Reference Card for GNU Emacs version 20 on Unix systems was
% translated into Polish language by W{\l}odek Bzyl (matwb@univ.gda.pl)
% who also added new section on `Dired' and added info about Polish
% support in Emacs to section `International Character Sets'.

% This file uses macros and fonts defined in the mex format.
% These macros and fonts are part of a current WEB2C
% distribution of TeX, for example teTeX (unix) fpTeX (windows).
% TeTeX comes with texconfig utility which could be used in
% particular to generate formats. Just run it and follow instructions.
%
% Note that the original Emacs refcard.tex uses macros and fonts
% defined in plain format. This file uses mex format which is
% a Polish adaptation of plain.

%**start of header

\ifx\MeX\undefined
  \errmessage{This file requires `mex' format to be typeset correctly.
    See head of this file for the comments how to generate mex format}
 \endinput
\else
  \prefixing
\fi

\newcount\columnsperpage

% This file can be printed with 1, or 2 columns per page (see below).
% Specify how many you want here.

\columnsperpage=2

% PDF output layout.  0 for A4, 1 for letter (US), a `l' is added for
% a landscape layout.

\input pdflayout.sty
\pdflayout=(0)

% Nothing else needs to be changed.
% Copyright (C) 1999, 2001, 2002, 2003, 2004, 2005,
%   2006, 2007  Free Software Foundation, Inc.

% This file is part of GNU Emacs.

% GNU Emacs is free software; you can redistribute it and/or modify
% it under the terms of the GNU General Public License as published by
% the Free Software Foundation; either version 3, or (at your option)
% any later version.

% GNU Emacs is distributed in the hope that it will be useful,
% but WITHOUT ANY WARRANTY; without even the implied warranty of
% MERCHANTABILITY or FITNESS FOR A PARTICULAR PURPOSE.  See the
% GNU General Public License for more details.

% You should have received a copy of the GNU General Public License
% along with GNU Emacs; see the file COPYING.  If not, write to
% the Free Software Foundation, Inc., 51 Franklin Street, Fifth Floor,
% Boston, MA 02110-1301, USA.

% This file is intended to be processed by plain TeX (TeX82).
%
% The final reference card has six columns, three on each side.
% This file can be used to produce it in any of three ways:
% 1 column per page
%    produces six separate pages, each of which needs to be reduced to 80%.
%    This gives the best resolution.
% 2 columns per page
%    produces three already-reduced pages.
%    You will still need to cut and paste.
% Which mode to use is controlled by setting \columnsperpage above.
%
% Author:
%  Stephen Gildea
%  Internet: gildea@stop.mail-abuse.org
%
% Thanks to Paul Rubin, Bob Chassell, Len Tower, and Richard Mlynarik
% for their many good ideas.

% If there were room, it would be nice to see a section on Dired.

\def\versionnumber{1.2}
\def\versionemacs{22}
\def\versiondate{czerwiec 2006} % latest update
\def\year{2007}                 % latest copyright year

\def\shortcopyrightnotice{\vskip 1ex plus 2 fill
  \centerline{\small \copyright\ \year\ Free Software Foundation, Inc.
  Permissions on back.  Version \versionnumber}}

\def\copyrightnotice{
\vskip 1ex plus 2 fill\begingroup\small
\centerline{Copyright \copyright\ \year\ Free Software Foundation, Inc.}
\centerline{Wersja \versionnumber{} dla GNU Emacsa \versionemacs,
  \versiondate}
\centerline{projekt Stephen Gildea}
\centerline{t/lumaczenie W/lodek Bzyl}

Permission is granted to make and distribute copies of
this card provided the copyright notice and this permission notice
are preserved on all copies.

For copies of the GNU Emacs manual, write to the Free Software
Foundation, Inc., 51 Franklin Street, Fifth Floor, Boston, MA  02110-1301  USA

\endgroup}

% make \bye not \outer so that the \def\bye in the \else clause below
% can be scanned without complaint.
\def\bye{\par\vfill\supereject\end}

\newdimen\intercolumnskip	%horizontal space between columns
\newbox\columna			%boxes to hold columns already built
\newbox\columnb

\def\ncolumns{\the\columnsperpage}

\message{[\ncolumns\space
  column\if 1\ncolumns\else s\fi\space per page]}

\def\scaledmag#1{ scaled \magstep #1}

% This multi-way format was designed by Stephen Gildea October 1986.
% Note that the 1-column format is fontfamily-independent.
\if 1\ncolumns			%one-column format uses normal size
  \hsize 4in
  \vsize 10in
%  \voffset -.7in
  \font\titlefont=\fontname\tenbf \scaledmag3
  \font\headingfont=\fontname\tenbf \scaledmag2
  \font\smallfont=\fontname\sevenrm
  \font\smallsy=\fontname\sevensy

  \footline{\hss\folio}
  \def\makefootline{\baselineskip10pt\hsize6.5in\line{\the\footline}}
\else				%2 or 3 columns uses prereduced size
  \hsize 3.2in
  \vsize 7.95in
%  \hoffset -.75in
%  \voffset -.745in
  \font\titlefont=plbx10 \scaledmag2
  \font\headingfont=plbx10 \scaledmag1
  \font\smallfont=plr6
  \font\smallsy=plsy6
  \font\eightrm=plr8
  \font\eightbf=plbx8
  \font\eightit=plti8
  \font\eighttt=pltt8
  \font\eightmi=plmi8
  \font\eightsy=plsy8
  \textfont0=\eightrm
  \textfont1=\eightmi
  \textfont2=\eightsy
  \def\rm{\eightrm}
  \def\bf{\eightbf}
  \def\it{\eightit}
  \def\tt{\eighttt}
  \normalbaselineskip=.8\normalbaselineskip
  \normallineskip=.8\normallineskip
  \normallineskiplimit=.8\normallineskiplimit
  \setbox\strutbox=\hbox{\vrule height6.5pt depth2.5pt width0pt}
  \normalbaselines\rm		%make definitions take effect

  \if 2\ncolumns
    \let\maxcolumn=b
    \footline{\hss\rm\folio\hss}
    \def\makefootline{\vskip 2in \hsize=6.86in\line{\the\footline}}
  \else
%    \errhelp{You must set \columnsperpage equal to 1, 2, or 3.}
%    \errmessage{Illegal number of columns per page}
    \errhelp{\columnsperpage powinna by/c r/owna 1 albo 2.}
    \errmessage{Niedozwolona liczba kolumn na stronie.}
  \fi

  \intercolumnskip=.46in
  \def\abc{a}
  \output={%			%see The TeXbook page 257
      % This next line is useful when designing the layout.
      %\immediate\write16{Column \folio\abc\space starts with \firstmark}
      \if \maxcolumn\abc \multicolumnformat \global\def\abc{a}
      \else\if a\abc
	\global\setbox\columna\columnbox \global\def\abc{b}
        %% in case we never use \columnb (two-column mode)
        \global\setbox\columnb\hbox to -\intercolumnskip{}
      \else
	\global\setbox\columnb\columnbox \global\def\abc{c}\fi\fi}
  \def\multicolumnformat{\shipout\vbox{\makeheadline
      \hbox{\box\columna\hskip\intercolumnskip
        \box\columnb\hskip\intercolumnskip\columnbox}
      \makefootline}\advancepageno}
  \def\columnbox{\leftline{\pagebody}}

  \def\bye{\par\vfill\supereject
    \if a\abc \else\null\vfill\eject\fi
    \if a\abc \else\null\vfill\eject\fi
    \end}
\fi

% we won't be using math mode much, so redefine some of the characters
% we might want to talk about
\catcode`\^=12
\catcode`\_=12

\chardef\\=`\\
\chardef\{=`\{
\chardef\}=`\}

%\hyphenation{mini-buf-fer}

\parindent 0pt
\parskip 1ex plus .5ex minus .5ex

\def\small{\smallfont\textfont2=\smallsy\baselineskip=.8\baselineskip}

% newcolumn - force a new column.  Use sparingly, probably only for
% the first column of a page, which should have a title anyway.
\outer\def\newcolumn{\vfill\eject}

% title - page title.  Argument is title text.
\outer\def\title#1{{\titlefont\centerline{#1}}\vskip 1ex plus .5ex}

% section - new major section.  Argument is section name.
\outer\def\section#1{\par\filbreak
  \vskip 3ex plus 2ex minus 2ex {\headingfont #1}\mark{#1}%
  \vskip 2ex plus 1ex minus 1.5ex}

\newdimen\keyindent

% beginindentedkeys...endindentedkeys - key definitions will be
% indented, but running text, typically used as headings to group
% definitions, will not.
\def\beginindentedkeys{\keyindent=1em}
\def\endindentedkeys{\keyindent=0em}
\endindentedkeys

% paralign - begin paragraph containing an alignment.
% If an \halign is entered while in vertical mode, a parskip is never
% inserted.  Using \paralign instead of \halign solves this problem.
\def\paralign{\vskip\parskip\halign}

% \<...> - surrounds a variable name in a code example
\def\<#1>{{\it #1\/}}

% kbd - argument is characters typed literally.  Like the Texinfo command.
\def\kbd#1{{\tt#1}\null}	%\null so not an abbrev even if period follows

% beginexample...endexample - surrounds literal text, such a code example.
% typeset in a typewriter font with line breaks preserved
\def\beginexample{\par\leavevmode\begingroup
  \obeylines\obeyspaces\parskip0pt\tt}
{\obeyspaces\global\let =\ }
\def\endexample{\endgroup}

% (WB) -- changed macros

% key - definition of a key.
% \key{description of key}{key-name}
% prints the description left-justified, and the key-name in a \kbd
% form near the right margin.
%\def\key#1#2{\leavevmode\hbox to \hsize{\vtop
%  {\hsize=.75\hsize\rightskip=1em
%  \hskip\keyindent\relax#1}\kbd{#2}\hfil}}
\def\key#1#2{\leavevmode\hbox to \hsize{\vbox
  {\hsize=.75\hsize\rightskip=1em
   \raggedright
   \hskip\keyindent\hangindent=1em\strut#1\strut\par}\kbd{\quad#2}\hss}}

\newbox\metaxbox
\setbox\metaxbox\hbox{\kbd{M-x }}
\newdimen\metaxwidth
\metaxwidth=\wd\metaxbox

% metax - definition of a M-x command.
% \metax{description of command}{M-x command-name}
% Tries to justify the beginning of the command name at the same place
% as \key starts the key name.  (The "M-x " sticks out to the left.)
%\def\metax#1#2{\leavevmode\hbox to \hsize{\hbox to .75\hsize
%  {\hskip\keyindent\relax#1\hfil}%
%  \hskip -\metaxwidth minus 1fil
%  \kbd{#2}\hfil}}
\def\metax#1#2{\leavevmode\hbox to \hsize{\vbox
  {\hsize=.74\hsize\rightskip=1em
   \raggedright
   \hskip\keyindent\hangindent=1em\strut#1\strut\par}%
   \hskip-\metaxwidth minus 1fil
   \kbd{#2}\hss}}

% threecol - like "key" but with two key names.
% for example, one for doing the action backward, and one for forward.
%\def\threecol#1#2#3{\hskip\keyindent\relax#1\hfil&\kbd{#2}\hfil\quad
%  &\kbd{#3}\hfill\quad\cr}
\def\threecol#1#2#3{\hskip\keyindent\relax#1\hfil&\kbd{#2}\hfil\quad
  &\kbd{#3}\hfill\cr}
\def\threecolheader#1#2#3{\threecol#1#2#3\noalign{\smallskip}}

% (WB) -- new macros

\newdimen\raggedstretch
\newskip\raggedparfill \raggedparfill=0pt plus 1fil
\def\nohyphens
   {\hyphenpenalty10000\exhyphenpenalty10000\pretolerance10000}
\def\raggedspaces
   {\spaceskip=0.3333em\relax
    \xspaceskip=0.5em\relax}
\def\raggedright
   {\raggedstretch=6em
    \nohyphens
    \rightskip=0pt plus \raggedstretch
    \raggedspaces
    \parfillskip=\raggedparfill
    \relax}
\def\newline{\hfil\break}

\hfuzz=3pt

%**end of header


%\title{GNU Emacs Reference Card}
\title{Przegl/ad polece/n GNU Emacsa}

\centerline{(dla wersji \versionemacs)}

\section{Uruchamianie Emacsa}

%To enter GNU Emacs 20, just type its name: \kbd{emacs}
Aby uruchomi/c GNU Emacsa \versionemacs, napisz jego nazw/e: \kbd{emacs}

%To read in a file to edit, see Files, below.
Aby wczyta/c plik do edycji, patrz rozdzia/l {\bf Pliki} poni/zej.

\section{Opuszczanie Emacsa}

%\key{suspend Emacs (or iconify it under X)}{C-z}
\key{tymczasowe zatrzymanie Emacsa}{C-z}
%\key{exit Emacs permanently}{C-x C-c}
\key{zako/nczenie sesji z Emacsem}{C-x C-c}

\section{Pliki}

\key{{\bf wczytaj} plik do Emacsa}{C-x C-f}
\key{{\bf zapisz} plik na dysk}{C-x C-s}
\key{zapisz {\bf wszystkie} pliki}{C-x s}
\key{{\bf wstaw} zawarto/s/c innego pliku do bufora}{C-x i}
%\key{replace this file with the file you really want}{C-x C-v}
\key{zamie/n plik w buforze na inny}{C-x C-v}
%\key{write buffer to a specified file}{C-x C-w}
\key{zapisz bufor do pliku z podaniem nazwy}{C-x C-w}
%\key{version control checkin//checkout}{C-x C-q}
\key{kontrola wersji pliku `checkin//checkout'}{C-x C-q}

%\section{Getting Help}
\section{Uzyskiwanie pomocy}

%The help system is simple.  Type \kbd{C-h} (or \kbd{F1}) and follow
%the directions.  If you are a first-time user, type \kbd{C-h t} for a
%{\bf tutorial}.
Napisz \kbd{C-h} (lub \kbd{F1}) i post/epuj
wed/lug dalszych instrukcji. Je/sli jeste/s pocz/atkuj/acym u/zytkownikiem,
napisz \kbd{C-u C-h t Polish} aby wywo/la/c {\bf samouczek} Emacsa
po polsku.

%\key{remove help window}{C-x 1}
%\key{scroll help window}{C-M-v}
\key{usu/n okno pomocy}{C-x 1}
\key{przewi/n okno pomocy}{C-M-v}

%\key{apropos: show commands matching a string}{C-h a}
%\key{show the function a key runs}{C-h c}
%\key{describe a function}{C-h f}
%\key{get mode-specific information}{C-h m}
\key{apropos: poka/z polecenia zgodne z napisem}{C-h a}
\key{poka/z funkcj/e uruchamian/a przez klawisz}{C-h c}
\key{opisz funkcj/e}{C-h f}
\key{poka/z informacj/e odnosz/ac/a si/e do trybu}{C-h m}


%\section{Error Recovery}
%\section{Powr/ot do sytuacji wyj/sciowej w przypadku b/l/ed/ow}
\section{Usuwanie b/l/ed/ow}

%\key{{\bf abort} partially typed or executing command}{C-g}
%\metax{{\bf recover} a file lost by a system crash}{M-x recover-file}
%\key{{\bf undo} an unwanted change}{C-x u {\rm or} C-_}
%\metax{restore a buffer to its original contents}{M-x revert-buffer}
%\key{redraw garbaged screen}{C-l}
\key{{\bf przerwij} cz/e/sciowo napisane lub\newline wykonywane polecenie}{C-g}
\metax{{\bf odzyskaj} plik zgubiony w wyniku\newline za/lamania systemu}
  {M-x recover-file}
\key{{\bf anuluj} niechcian/a zmian/e}{C-x u {\rm lub} C-_}
\metax{wczytaj plik wg aktualnej zawarto/sci na dysku}{M-x revert-buffer}
\key{uporz/adkuj za/smiecony ekran}{C-l}

\shortcopyrightnotice

%\section{Incremental Search}
\section{Szukanie przyrostowe}

%\key{search forward}{C-s}
%\key{search backward}{C-r}
%\key{regular expression search}{C-M-s}
%\key{reverse regular expression search}{C-M-r}
\key{szukaj wprz/od//wstecz ({\tt C-f} aby zako/nczy/c)}{C-s//C-r}
\key{szukaj wprz/od tekstu zgodnego z~wpisywanym wyra/zeniem regularnym}{C-M-s}
\key{szukaj wstecz tekstu zgodnego z~wpisywanym wyra/zeniem regularnym}{C-M-r}

%\key{select previous search string}{M-p}
%\key{select next later search string}{M-n}
%\key{exit incremental search}{RET}
%\key{undo effect of last character}{DEL}
%\key{abort current search}{C-g}
\key{wybierz poprzedni napis}{M-p}
\key{wybierz nast/epny napis}{M-n}
\key{zako/ncz szukanie przyrostowe}{RET}
\key{anuluj rezultat ostatniej poprawki}{DEL}
\key{przerwij szukanie}{C-g}

%Use \kbd{C-s} or \kbd{C-r} again to repeat the search in either direction.
%If Emacs is still searching, \kbd{C-g} cancels only the part not done.
Ponowne \kbd{C-s}//\kbd{C-r} powtarza szukanie wprz/od//wstecz.
%If Emacs is still searching, \kbd{C-g} cancels only the part not done.
% Patrz wyja/snienie powy/zej.

\section{Przemieszczanie kursora}

\paralign to \hsize{#\tabskip=10pt plus 1 fil&#\tabskip=0pt&#\cr
%\threecol{{\bf entity to move over}}{{\bf backward}}{{\bf forward}}
%\threecol{character}{C-b}{C-f}
%\threecol{word}{M-b}{M-f}
%\threecol{line}{C-p}{C-n}
%\threecol{go to line beginning (or end)}{C-a}{C-e}
%\threecol{sentence}{M-a}{M-e}
%\threecol{paragraph}{M-\{}{M-\}}
%\threecol{page}{C-x [}{C-x ]}
%\threecol{sexp}{C-M-b}{C-M-f}
%\threecol{function}{C-M-a}{C-M-e}
%\threecol{go to buffer beginning (or end)}{M-<}{M->}
\threecolheader{{\bf przemie/s/c kursor}}{{\bf wstecz}}{{\bf wprz/od}}
\threecol{o znak}{C-b}{C-f}
\threecol{o s/lowo}{M-b}{M-f}
\threecol{o lini/e wy/zej//ni/zej}{C-p}{C-n}
\threecol{na pocz/atek//koniec linii}{C-a}{C-e}
\threecol{o zdanie}{M-a}{M-e}
\threecol{o akapit}{M-\{}{M-\}}
\threecol{o stron/e}{C-x [}{C-x ]}
\threecol{o s-wyra/zenie}{C-M-b}{C-M-f}
\threecol{na pocz/atek//koniec funkcji}{C-M-a}{C-M-e}
\threecol{na pocz/atek//koniec bufora}{M-<}{M->}
}

%\key{scroll to next screen}{C-v}
%\key{scroll to previous screen}{M-v}
%\key{scroll left}{C-x <}
%\key{scroll right}{C-x >}
%\key{scroll current line to center of screen}{C-u C-l}
\key{przewi/n do nast/epnego ekranu}{C-v}
\key{przewi/n do poprzedniego ekranu}{M-v}
\key{przewi/n w lewo}{C-x <}
\key{przewi/n w prawo}{C-x >}
\key{umie/s/c lini/e z kursorem na /srodku ekranu}{C-u C-l}


%\section{Killing and Deleting}
\section{Kasowanie i usuwanie}

\paralign to \hsize{#\tabskip=10pt plus 1 fil&#\tabskip=0pt&#\cr
%\threecol{{\bf entity to kill}}{{\bf backward}}{{\bf forward}}
%\threecol{character (delete, not kill)}{DEL}{C-d}
%\threecol{word}{M-DEL}{M-d}
%\threecol{line (to end of)}{M-0 C-k}{C-k}
%\threecol{sentence}{C-x DEL}{M-k}
%\threecol{sexp}{M-- C-M-k}{C-M-k}
\threecolheader{{\bf obiekt do skasowania}}{{\bf wstecz}}{{\bf wprz/od}}
\threecol{znak (usu/n, nie kasuj)}{DEL}{C-d}
\threecol{s/lowo}{M-DEL}{M-d}
\threecol{linia (od kursora do ko/nca)}{M-0 C-k}{C-k}
\threecol{zdanie}{C-x DEL}{M-k}
\threecol{s-wyra/zenie}{M-- C-M-k}{C-M-k}
}

%\key{kill {\bf region}}{C-w}
%\key{copy region to kill ring}{M-w}
%\key{kill through next occurrence of {\it char}}{M-z {\it char}}
\key{kasuj obszar}{C-w}
\key{wstaw obszar do `kill ring'}{M-w}
\key{kasuj wszystko a/z do wyst/apienia {\it char}}{M-z {\it char}}

%\key{yank back last thing killed}{C-y}
%\key{replace last yank with previous kill}{M-y}
\key{wstaw ostatnio skasowany obiekt}{C-y}
\key{zamie/n wstawiony obiekt z uprzednio skasowanym}{M-y}

%\section{Marking}
\section{Zaznaczanie}

%\key{set mark here}{C-@ {\rm or} C-SPC}
%\key{exchange point and mark}{C-x C-x}
\key{wstaw znacznik w pozycji kursora}{C-@ {\rm or} C-SPC}
\key{zamie/n pozycje kursora i znacznika}{C-x C-x}

%\key{set mark {\it arg\/} {\bf words} away}{M-@}
%\key{mark {\bf paragraph}}{M-h}
%\key{mark {\bf page}}{C-x C-p}
%\key{mark {\bf sexp}}{C-M-@}
%\key{mark {\bf function}}{C-M-h}
%\key{mark entire {\bf buffer}}{C-x h}
\key{zaznacz s/lowo po {\it arg\/} s/l/ow}{M-@}
\key{zaznacz akapit}{M-h}
\key{zaznacz stron/e}{C-x C-p}
\key{zaznacz s-wyra/zenie}{C-M-@}
\key{zaznacz funkcj/e}{C-M-h}
\key{zaznacz ca/ly bufor}{C-x h}

%\section{Query Replace}
\section{Zamiana z zapytaniem}

%\key{interactively replace a text string}{M-\%}
%\metax{using regular expressions}{M-x query-replace-regexp}
\key{zamiana tekstu w trybie interakcyjnym}{M-\%}
\key{z u/zyciem wyra/ze/n regularnych}{C-M-\%}

%Valid responses in query-replace mode are
Odpowiedzi w interakcyjnym trybie zamiany:

%\key{{\bf replace} this one, go on to next}{SPC}
%\key{replace this one, don't move}{,}
%\key{{\bf skip} to next without replacing}{DEL}
%\key{replace all remaining matches}{!}
%\key{{\bf back up} to the previous match}{^}
%\key{{\bf exit} query-replace}{RET}
%\key{enter recursive edit (\kbd{C-M-c} to exit)}{C-r}
\key{{\bf zamie/n} i wyszukaj nast/epny tekst}{SPC}
\key{zamie/n nie przemieszczaj/ac kursora}{,}
\key{{\bf nie zamieniaj} i wyszukaj nast/epny tekst}{DEL}
\key{wyszukaj i zamie/n wszystkie pozosta/le teksty}{!}
\key{{\bf powr/o/c} do poprzedniej zamiany}{^}
\key{{\bf zako/ncz} zamian/e interakcyjn/a}{RET}
\key{wejd/x do trybu rekursywnej edycji (\kbd{C-M-c} aby zako/nczy/c)}{C-r}

%\section{Multiple Windows}
\section{Wiele okien}

%When two commands are shown, the second is for ``other frame.''
Drugie z polece/n dotyczy ,,innej ramki''

%\key{delete all other windows}{C-x 1}
\key{usu/n pozosta/le okna}{C-x 1}

{\setbox0=\hbox{\kbd{0}}\advance\hsize by 0\wd0
\paralign to \hsize{#\tabskip=10pt plus 1 fil&#\tabskip=0pt&#\cr
%\threecol{split window, above and below}{C-x 2\ \ \ \ }{C-x 5 2}
%\threecol{delete this window}{C-x 0\ \ \ \ }{C-x 5 0}
\threecol{podziel okno (jedno nad drugim)}{C-x 2\ \ \ \ }{C-x 5 2}
\threecol{usu/n okno}{C-x 0\ \ \ \ }{C-x 5 0}
}}
%\key{split window, side by side}{C-x 3}
\key{podziel okno (jedno obok drugiego)}{C-x 3}

\key{przewijaj w innym oknie}{C-M-v}

{\setbox0=\hbox{\kbd{0}}\advance\hsize by 2\wd0
\paralign to \hsize{#\tabskip=10pt plus 1 fil&#\tabskip=0pt&#\cr
%\threecol{switch cursor to another window}{C-x o}{C-x 5 o}
%\threecol{select buffer in other window}{C-x 4 b}{C-x 5 b}
%\threecol{display buffer in other window}{C-x 4 C-o}{C-x 5 C-o}
%\threecol{find file in other window}{C-x 4 f}{C-x 5 f}
%\threecol{find file read-only in other window}{C-x 4 r}{C-x 5 r}
%\threecol{run Dired in other window}{C-x 4 d}{C-x 5 d}
%\threecol{find tag in other window}{C-x 4 .}{C-x 5 .}
\threecol{przenie/s kursor do innego okna}{C-x o}{C-x 5 o}
\threecol{wybierz bufor w innym oknie}{C-x 4 b}{C-x 5 b}
\threecol{poka/z bufor w innym oknie}{C-x 4 C-o}{C-x 5 C-o}
\threecol{znajd/x plik i poka/z go w innym oknie}{C-x 4 f}{C-x 5 f}
\threecol{jak wy/zej, tylko w trybie do czytania}{C-x 4 r}{C-x 5 r}
\threecol{uruchom `Dired' w innym oknie}{C-x 4 d}{C-x 5 d}
\threecol{znajd/x definicj/e w innym oknie}{C-x 4 .}{C-x 5 .}
}}

%\key{grow window taller}{C-x ^}
%\key{shrink window narrower}{C-x \{}
%\key{grow window wider}{C-x \}}
\key{powi/eksz okno w pionie}{C-x ^}
\key{zmniejsz okno w poziomie}{C-x \{}
\key{poszerz okno}{C-x \}}

%\section{Formatting}
\section{Formatowanie}

%\key{indent current {\bf line} (mode-dependent)}{TAB}
%\key{indent {\bf region} (mode-dependent)}{C-M-\\}
%\key{indent {\bf sexp} (mode-dependent)}{C-M-q}
%\key{indent region rigidly {\it arg\/} columns}{C-x TAB}
\key{wetnij bie/z/ac/a {\bf lini/e} (zale/zne od trybu)}{TAB}
\key{wetnij {\bf obszar} (zale/zne od trybu)}{C-M-\\}
\key{wetnij {\bf s-wyra/zenie} (zale/zne od trybu)}{C-M-q}
\key{wetnij obszar o {\it arg\/} kolumn}{C-x TAB}

%\key{insert newline after point}{C-o}
%\key{move rest of line vertically down}{C-M-o}
%\key{delete blank lines around point}{C-x C-o}
%\key{join line with previous (with arg, next)}{M-^}
%\key{delete all white space around point}{M-\\}
%\key{put exactly one space at point}{M-SPC}
\key{wstaw now/a lini/e za kursorem}{C-o}
\key{przesu/n cz/e/s/c linii za kursorem w d/o/l }{C-M-o}
\key{usu/n puste linie wok/o/l kursora}{C-x C-o}
\key{po/l/acz lini/e z poprzedni/a (z {\it arg\/} -- z nast/epn/a)}{M-^}
\key{usu/n odst/epy dooko/la kursora}{M-\\}
\key{pozostaw dok/ladnie jedn/a spacj/e w pozycji kursora}{M-SPC}

%\key{fill paragraph}{M-q}
%\key{set fill column}{C-x f}
%\key{set prefix each line starts with}{C-x .}
\key{wype/lnij akapit}{M-q}
\key{ustaw numer kolumny dla trybu wype/lniania}{C-x f}
\key{ustaw przedrostek dla ka/zdego nowego wiersza}{C-x .}

%\key{set face}{M-g}
\key{ustaw czcionk/e}{M-g}

%\section{Case Change}
\section{Zamiana wielko/sci liter}

%\key{uppercase word}{M-u}
%\key{lowercase word}{M-l}
%\key{capitalize word}{M-c}
\key{zamie/n w s/lowie litery ma/le na du/ze}{M-u}
\key{zamie/n w s/lowie litery du/ze na ma/le}{M-l}
\key{zamie/n pierwsz/a liter/e w s/lowie na du/z/a}{M-c}

%\key{uppercase region}{C-x C-u}
%\key{lowercase region}{C-x C-l}
\key{zamie/n w obszarze litery ma/le na du/ze}{C-x C-u}
\key{zamie/n w obszarze litery du/ze na ma/le}{C-x C-l}

%\section{The Minibuffer}
\section{Minibufor}

%The following keys are defined in the minibuffer.
% Inne te/z s/a maj/a przypisane funkcje..

%\key{complete as much as possible}{TAB}
%\key{complete up to one word}{SPC}
%\key{complete and execute}{RET}
%\key{show possible completions}{?}
%\key{fetch previous minibuffer input}{M-p}
%\key{fetch later minibuffer input or default}{M-n}
%\key{regexp search backward through history}{M-r}
%\key{regexp search forward through history}{M-s}
%\key{abort command}{C-g}
\key{uzupe/lnij tekst o tyle o ile jest to mo/zliwe}{TAB}
\key{uzupe/lnij o jedno s/lowo}{SPC}
\key{uzupe/lnij i wykonaj}{RET}
\key{poka/z mo/zliwe uzupe/lnienia}{?}
\key{przywo/laj uprzednio wprowadzony tekst do minibufora}{M-p}
\key{przywo/laj nast/epny tekst z~`kill ring' do minibufora}{M-n}
\key{wyszukuj wstecz poprzez histori/e wprowadze/n}{M-r}
\key{wyszukuj wprz/od poprzez histori/e wprowadze/n}{M-s}
\key{przerwij wykonywane polecenie}{C-g}

%Type \kbd{C-x ESC ESC} to edit and repeat the last command that used the
%minibuffer.  Type \kbd{F10} to activate the menu bar using the minibuffer.
Napisz \kbd{C-x ESC ESC} aby poprawia/c i wykona/c polecenie,
kt/ore ostatnio u/zywa/lo minibufora.
Napisz \kbd{F10} aby uaktywni/c menu w minibuforze.


\newcolumn
%\title{GNU Emacs Reference Card}
\title{Przegl/ad polece/n GNU Emacsa}

\section{Bufory}

%\key{select another buffer}{C-x b}
%\key{list all buffers}{C-x C-b}
%\key{kill a buffer}{C-x k}
\key{wybierz inny bufor}{C-x b}
\key{poka/z spis wszystkich bufor/ow}{C-x C-b}
\key{skasuj bufor}{C-x k}

%\section{Transposing}
\section{Przestawianie}

%\key{transpose {\bf characters}}{C-t}
%\key{transpose {\bf words}}{M-t}
%\key{transpose {\bf lines}}{C-x C-t}
%\key{transpose {\bf sexps}}{C-M-t}
\key{przestaw {\bf znaki}}{C-t}
\key{przestaw {\bf s/lowa}}{M-t}
\key{przestaw {\bf linie}}{C-x C-t}
\key{przestaw {\bf s-wyra/zenia}}{C-M-t}

% Removed -- there is no Polish disctionary for ispell.
%\section{Spelling Check}
%
%\key{check spelling of current word}{M-\$}
%\metax{check spelling of all words in region}{M-x ispell-region}
%\metax{check spelling of entire buffer}{M-x ispell-buffer}

%\section{Tags}
\section{Tags}

%\key{find a tag (a definition)}{M-.}
%\key{find next occurrence of tag}{C-u M-.}
%\metax{specify a new tags file}{M-x visit-tags-table}
\key{znajd/x okre/slenie (definicj/e)}{M-.}
\key{znajd/x nast/epne wyst/apienie definicji}{C-u M-.}
\metax{podaj nowy plik TAGS}{M-x visit-tags-table}

%\metax{regexp search on all files in tags table}{M-x tags-search}
%\metax{run query-replace on all the files}{M-x tags-query-replace}
%\key{continue last tags search or query-replace}{M-,}
\metax{wyszukiwanie tekstu zgodnego z~podanym wyra/zeniem regularnym
  we wszystkich plikach wymienionych w~TAGS}{M-x tags-search}
\metax{zamiana z zapytaniem we wszystkich\newline
  plikach wymienionych w~TAGS}{M-x tags-query-replace}
\key{kontynuuj wyszukiwanie lub zamian/e z~zapytaniem
  w~plikach wymienionych w~TAGS}{M-,}

%\section{Shells}
\section{Pow/loki}

%\key{execute a shell command}{M-!}
%\key{run a shell command on the region}{M-|}
%\key{filter region through a shell command}{C-u M-|}
%\key{start a shell in window \kbd{*shell*}}{M-x shell}
\key{wykonaj polecenie pow/loki}{M-!}
\key{wykonaj polecenie pow/loki na obszarze}{M-|}
\key{filtruj obszar poprzez polecenie pow/loki}{C-u M-|}
\key{uruchom pow/lok/e w oknie  \kbd{*shell*}}{M-x shell}

%\section{Rectangles}
\section{Prostok/aty}

%\key{copy rectangle to register}{C-x r r}
%\key{kill rectangle}{C-x r k}
%\key{yank rectangle}{C-x r y}
%\key{open rectangle, shifting text right}{C-x r o}
%\key{blank out rectangle}{C-x r c}
%\key{prefix each line with a string}{C-x r t}
\key{zapisz prostok/at do rejestru}{C-x r r}
\key{skasuj prostok/at}{C-x r k}
\key{wklej prostok/at}{C-x r y}
\key{wstaw pusty prostok/at przesuwaj/ac\newline tekst w~prawo}{C-x r o}
\key{wyczy/s/c prostok/at}{C-x r c}
\key{wstaw napis na pocz/atku ka/zdej linii}{C-x r t}

%\section{Abbrevs}
\section{Skr/oty}

%\key{add global abbrev}{C-x a g}
%\key{add mode-local abbrev}{C-x a l}
%\key{add global expansion for this abbrev}{C-x a i g}
%\key{add mode-local expansion for this abbrev}{C-x a i l}
%\key{explicitly expand abbrev}{C-x a e}
\key{dodaj skr/ot globalnie}{C-x a g}
\key{dodaj skr/ot lokalny dla trybu}{C-x a l}
\key{dodaj rozwini/ecie globalne dla skr/otu}{C-x a i g}
\key{dodaj rozwini/ecie lokalne dla trybu dla skr/otu}{C-x a i l}
\key{rozwi/n teraz skr/ot}{C-x a e}

%\key{expand previous word dynamically}{M-//}
\key{uzupe/lnij dynamicznie poprzednie s/lowo}{M-//}

%\section{Regular Expressions}
\section{Wyra/zenia regularne}

%\key{any single character except a newline}{. {\rm(dot)}}
%\key{zero or more repeats}{*}
%\key{one or more repeats}{+}
%\key{zero or one repeat}{?}
%\key{quote regular expression special character {\it c\/}}{\\{\it c}}
%\key{alternative (``or'')}{\\|}
%\key{grouping}{\\( {\rm$\ldots$} \\)}
%\key{same text as {\it n\/}th group}{\\{\it n}}
%\key{at word break}{\\b}
%\key{not at word break}{\\B}
\key{dowolny znak za wyj/atkiem znaku nowej linii}{. {\rm(kropka)}}
\key{zero lub wi/ecej powt/orze/n}{*}
\key{jedno lub wi/ecej powt/orze/n}{+}
\key{zero lub jedno powt/orzenie}{?}
\key{traktuj dos/lownie nast/epny znak
  (nawet je/sli znak ma specjalne znaczenie) {\it c\/}}{\\{\it c}}
\key{alternatywa (`lub')}{\\|}
\key{grupowanie}{\\( {\rm$\ldots$} \\)}
\key{tekst n-tej grupy}{\\{\it n}}
\key{na pocz/atku lub ko/ncu s/lowa}{\\b}
\key{nie na pocz/atku i nie na ko/ncu s/lowa}{\\B}

\paralign to \hsize{#\tabskip=10pt plus 1 fil&#\tabskip=0pt&#\cr
%\threecol{{\bf entity}}{{\bf match start}}{{\bf match end}}
%\threecol{line}{^}{\$}
%\threecol{word}{\\<}{\\>}
%\threecol{buffer}{\\`}{\\'}
\threecolheader{{\bf obiekt do por/ownania}}{{\bf pocz/atek}}{{\bf koniec}}
\threecol{linia}{^}{\$}
\threecol{s/lowo}{\\<}{\\>}
\threecol{bufor}{\\`}{\\'}}
%\threecol{{\bf class of characters}}{{\bf match these}}{{\bf match others}}
%\threecol{explicit set}{[ {\rm$\ldots$} ]}{[^ {\rm$\ldots$} ]}
%\threecol{word-syntax character}{\\w}{\\W}
%\threecol{character with syntax {\it c}}{\\s{\it c}}{\\S{\it c}}
\paralign to \hsize{#\tabskip=10pt plus 1 fil&#\tabskip=0pt&#\cr
\threecolheader{{\bf kategoria znak/ow}}{{\bf por/ownaj z}}
  {{\bf \hbox to 0pt{pozosta/lymi\hss}}}
\threecol{podany zbi/or znak/ow}{[ {\rm$\ldots$} ]}{[^ {\rm$\ldots$} ]}
\threecol{znak kategorii sk/ladniowej `s/lowo'}{\\w}{\\W}
\threecol{znak kategorii sk/ladniowej {\it c}}{\\s{\it c}}{\\S{\it c}}
}

%\section{International Character Sets}
\section{Mi/edzynarodowe zestawy znak/ow}

%\metax{specify principal language}{M-x set-language-environment}
%\metax{show all input methods}{M-x list-input-methods}
%\key{enable or disable input method}{C-\\}
%\key{set coding system for next command}{C-x RET c}
%\metax{show all coding systems}{M-x list-coding-systems}
%\metax{choose preferred coding system}{M-x prefer-coding-system}
\metax{ustal g/l/owny j/ezyk}{M-x set-language-environment}
\metax{wypisz wszystkie metody wprowadzania znak/ow}{M-x list-input-methods}
\key{w/l/acz//wy/l/acz metod/e wprowadzania znak/ow}{C-\\}
\key{ustal system kodowania dla nast/epnego polecenia}{C-x RET c}
\metax{wypisz wszystkie systemy kodowania}{M-x list-coding-systems}
\metax{wybierz preferowany system\newline kodowania}{M-x prefer-coding-system}
\metax{wybierz metod/e wprowadzania znak/ow}{C-x RET C-\\}

Po wybraniu metody wprowadzania znak/ow {\tt polish-slash},\newline
ka/zd/a z~liter {\tt /a/c/e/l/n/o/s/x/z//} uzyskamy pisz/ac ciach `{\tt //}'
i~odpowiedni znak z~{\tt acelnosxz//}.

%\section{Registers}
\section{Rejestry}

%\key{save region in register}{C-x r s}
%\key{insert register contents into buffer}{C-x r i}
\key{zapisz obszar do rejestru}{C-x r s}
\key{wstaw zawarto/s/c rejestru do bufora}{C-x r i}

%\key{save value of point in register}{C-x r SPC}
%\key{jump to point saved in register}{C-x r j}
\key{zapisz pozycj/e kursora w~rejestrze}{C-x r SPC}
\key{przemie/s/c kursor do pozycji zapisanej w~rejestrze}{C-x r j}

%\section{Keyboard Macros}
\section{Makropolecenia}

%\key{{\bf start} defining a keyboard macro}{C-x (}
%\key{{\bf end} keyboard macro definition}{C-x )}
%\key{{\bf execute} last-defined keyboard macro}{C-x e}
%\key{append to last keyboard macro}{C-u C-x (}
%\metax{name last keyboard macro}{M-x name-last-kbd-macro}
%\metax{insert Lisp definition in buffer}{M-x insert-kbd-macro}
\key{{\bf zacznij} definicj/e makropolecenia}{C-x (}
\key{{\bf zako/ncz} definicj/e makropolecenia}{C-x )}
\key{{\bf wykonaj} ostatnio zdefiniowane makropolecenie}{C-x e}
\key{dopisz do definicji ostatniego makropolecenia}{C-u C-x (}
\metax{nazwij ostatnie makropolecenie}{M-x name-last-kbd-macro}
\metax{wpisz do bufora nazwane\newline makro Lispowe}{M-x insert-kbd-macro}

%\section{Info}
\section{Info}

%\key{enter the Info documentation reader}{C-h i}
%\key{find specified function or variable in Info}{C-h S}
\key{wejd/x w tryb czytania dokumentacji Info}{C-h i}
\key{wyszukaj podan/a funkcj/e lub zmienn/a w Info}{C-h S}
\beginindentedkeys

%Moving within a node:
Poruszanie si/e w obr/ebie w/ez/la Info:

%\key{scroll forward}{SPC}
%\key{scroll reverse}{DEL}
%\key{beginning of node}{. {\rm (dot)}}
\key{przegl/adaj do przodu}{SPC}
\key{przegl/adaj do ty/lu}{DEL}
\key{na pocz/atek w/ez/la}{. {\rm (kropka)}}

%Moving between nodes:
Poruszanie si/e pomi/edzy w/ez/lami:

%\key{{\bf next} node}{n}
%\key{{\bf previous} node}{p}
%\key{move {\bf up}}{u}
%\key{select menu item by name}{m}
%\key{select {\it n\/}th menu item by number (1--9)}{{\it n}}
%\key{follow cross reference  (return with \kbd{l})}{f}
%\key{return to last node you saw}{l}
%\key{return to directory node}{d}
%\key{go to any node by name}{g}
\key{{\bf nast/epny} w/eze/l}{n}
\key{{\bf poprzedni} w/eze/l}{p}
\key{przenie/s si/e {\bf wy/zej}}{u}
\key{wybierz pozycj/e z menu, podaj/ac jej nazw/e}{m}
\key{wybierz n-t/a pozycj/e z menu podaj/ac liczb/e~(1-9)}{{\it n}}
\key{sprawd/x odsy/lacz (powr/ot z \kbd{l})}{f}
\key{powr/o/c do ostatnio przegl/adanego w/ez/la}{l}
\key{powr/o/c do skorowidza}{d}
\key{wybierz w/eze/l podaj/ac jego nazw/e}{g}

%Other:
Pozosta/le polecenia:

%\key{run Info {\bf tutorial}}{h}
%\key{{\bf quit} Info}{q}
%\key{search nodes for regexp}{M-s}
\key{wywo/laj {\bf samouczek} Info}{h}
\key{wyszukaj zagadnienie w indeksach}{i}
\key{wyszukuj w~w/ez/lach tekst zgodny\newline
  z~podanym wyra/zeniem regularnym}{s}
\key{{\bf zako/ncz} Info}{q}

\endindentedkeys

%\section{Dired, the Directory Editor}
\section{Dired -- edytor katalog/ow}

\key{wywo/lanie edytora katalog/ow}{C-x d}
\key{ustaw flag/e `D' (do usuni/ecia) na pliku}{d}
\key{ustaw flag/e `D' na plikach zapasowych}{\~{}}
\key{zdejmij flag/e `D' z pliku}{u}
\key{usu/n pliki oznaczone flag/a `D'}{x}
\key{uaktualnij zawarto/s/c bufora}{g}
\key{wczytaj plik wskazywany przez kursor do bufora}{f}
\key{prze/l/acz mi/edzy porz/adkiem alfabetycznym a~porz/adkiem wed/lug
  daty i~czasu powstania pliku}{s}

\metax{wybierz z~bie/z/acego katalogu i~jego podkatalog/ow wszystkie pliki,
  kt/ore zawieraj/a tekst zgodny z~podanym wyra/zeniem regularnym}
  {M-x find-grep-dired}

%\section{Commands Dealing with Emacs Lisp}
\section{Polecenia dotycz/ace j/ezyka Emacs Lisp}

%\key{eval {\bf sexp} before point}{C-x C-e}
%\key{eval current {\bf defun}}{C-M-x}
%\metax{eval {\bf region}}{M-x eval-region}
%\key{read and eval minibuffer}{M-:}
%\metax{load from standard system directory}{M-x load-library}
\key{oblicz {\bf s-wyra/zenie} przed kursorem}{C-x C-e}
\key{oblicz aktywn/a {\bf defun}}{C-M-x}
\metax{oblicz s-wyra/zenia w {\bf obszarze}}{M-x eval-region}
\key{wczytaj {\bf s-wyra/zenie} i oblicz je w~minibuforze}{M-:}
\metax{wczytaj bibliotek/e z~katalogu\newline systemowego}{M-x load-library}

%\section{Simple Customization}
\section{Proste modyfikacje}

%\metax{customize variables and faces}{M-x customize}
\metax{modyfikowanie warto/sci zmiennych\newline i czcionek}{M-x customize}

% The intended audience here is the person who wants to make simple
% customizations and knows Lisp syntax.

%Making global key bindings in Emacs Lisp (examples):
Przyk/lady przypisania klawisza funkcji Emacs Lispu\newline
z~wykorzystaniem notacji \kbd{"..."} i~notacji \kbd{[...]}:

\beginexample%
(global-set-key "\\C-cg" 'goto-line)
(global-set-key "\\M-\#" 'query-replace-regexp)
\smallskip
(global-set-key [?\\C-c ?g] 'goto-line)
(global-set-key [?\\M-\#] 'query-replace-regexp)
\endexample

%\section{Writing Commands}
\section{Tworzenie nowych polece/n}

\beginexample%
(defun \<nazwa-funkcji> (\<argumenty>)
  "\<dokumentacja>"
  (interactive "\<wzorzec>")
  \<tre/s/c funkcji>)
\endexample

%An example:
Przyk/lad:

\beginexample%
(defun this-line-to-top-of-window (line)
%  "Reposition line point is on to top of window.
%With ARG, put point on line ARG."
  "Przewi/n lini/e z kursorem do pierwszej linii okna.
Z ARG, przewi/n do linii ARG."
  (interactive "P")
  (recenter (if (null line)
                0
              (prefix-numeric-value line))))
\endexample

%The \kbd{interactive} spec says how to read arguments interactively.
%Type \kbd{C-h f interactive} for more details.
W funkcji \kbd{interactive} {\it wzorzec\/} opisuje, jak b/ed/a czytane
argumenty w trybie interakcyjnym. Szczeg/o/lowy opis uzyskasz
przez wywo/lanie \kbd{C-h f interactive}.

\copyrightnotice

\bye

% Local variables:
% compile-command: "pdftex pl-refcard"
% End:

% arch-tag: 19d15a16-70be-40c8-ad91-88899aac32a9
