% Reference Card for GNU Emacs version 22 on Unix systems
%**start of header
\newcount\columnsperpage
\newcount\letterpaper

% This file can be printed with 1, 2, or 3 columns per page (see below).
% Specify how many you want here.

\columnsperpage=3

% Set letterpaper to 0 for A4 paper, 1 for letter (US) paper.  Useful
% only when columnsperpage is 2 or 3.

\letterpaper=1

% Nothing else needs to be changed below this line.
% Copyright (C) 1987, 1993, 1996, 1997, 2001, 2002, 2003, 2004,
%   2005, 2006, 2007  Free Software Foundation, Inc.

% This file is part of GNU Emacs.

% GNU Emacs is free software; you can redistribute it and/or modify
% it under the terms of the GNU General Public License as published by
% the Free Software Foundation; either version 2, or (at your option)
% any later version.

% GNU Emacs is distributed in the hope that it will be useful,
% but WITHOUT ANY WARRANTY; without even the implied warranty of
% MERCHANTABILITY or FITNESS FOR A PARTICULAR PURPOSE.  See the
% GNU General Public License for more details.

% You should have received a copy of the GNU General Public License
% along with GNU Emacs; see the file COPYING.  If not, write to
% the Free Software Foundation, Inc., 51 Franklin Street, Fifth Floor,
% Boston, MA 02110-1301, USA.

% This file is intended to be processed by plain TeX (TeX82).
%
% The final reference card has six columns, three on each side.
% This file can be used to produce it in any of three ways:
% 1 column per page
%    produces six separate pages, each of which needs to be reduced to 80%.
%    This gives the best resolution.
% 2 columns per page
%    produces three already-reduced pages.
%    You will still need to cut and paste.
% 3 columns per page
%    produces two pages which must be printed sideways to make a
%    ready-to-use 8.5 x 11 inch reference card.
%    For this you need a dvi device driver that can print sideways.
% Which mode to use is controlled by setting \columnsperpage above.
%
% To compile and print this document:
% tex fr-refcard.tex
% dvips -t landscape fr-refcard.dvi
%

% Author:
%  Stephen Gildea
%  Internet: gildea@stop.mail-abuse.org
%
% Thanks to Paul Rubin, Bob Chassell, Len Tower, and Richard Mlynarik
% for their many good ideas.

% If there were room, it would be nice to see a section on Dired.

\def\versionnumber{2.3}
\def\versionemacs{22}
\def\versionyear{2006}          % latest update
\def\year{2007}                 % latest copyright year

\def\shortcopyrightnotice{\vskip 1ex plus 2 fill
  \centerline{\small \copyright\ \year\ Free Software Foundation, Inc.
  Permissions au dos.  v\versionnumber}}

\def\copyrightnotice{
\vskip 1ex plus 2 fill\begingroup\small
\centerline{Copyright \copyright\ \year\ Free Software Foundation, Inc.}
\centerline{v\versionnumber{} pour GNU Emacs version \versionemacs,
  \versionyear}
\centerline{conception de Stephen Gildea}
\centerline{traduction fran\c{c}aise d'\'Eric Jacoboni}

Vous pouvez faire et distribuer des copies de cette carte, pourvu que
la note de copyright et cette note de permission soient conserv\'ees sur
toutes les copies.

Pour les copies du manuel GNU Emacs, \'ecrivez \`a la Free Software
Foundation, Inc., 51 Franklin Street, Fifth Floor, Boston, MA  02110-1301  USA.

\endgroup}

% make \bye not \outer so that the \def\bye in the \else clause below
% can be scanned without complaint.
\def\bye{\par\vfill\supereject\end}

\newdimen\intercolumnskip       %horizontal space between columns
\newbox\columna                 %boxes to hold columns already built
\newbox\columnb

\def\ncolumns{\the\columnsperpage}

\message{[\ncolumns\space
  column\if 1\ncolumns\else s\fi\space per page]}

\def\scaledmag#1{ scaled \magstep #1}

% This multi-way format was designed by Stephen Gildea October 1986.
% Note that the 1-column format is fontfamily-independent.
\if 1\ncolumns                  %one-column format uses normal size
  \hsize 4in
  \vsize 10in
  \voffset -.7in
  \font\titlefont=\fontname\tenbf \scaledmag3
  \font\headingfont=\fontname\tenbf \scaledmag2
  \font\smallfont=\fontname\sevenrm
  \font\smallsy=\fontname\sevensy

  \footline{\hss\folio}
  \def\makefootline{\baselineskip10pt\hsize6.5in\line{\the\footline}}
\else                           %2 or 3 columns uses prereduced size
  \hsize 3.2in
  \vsize 7.95in
  \if 1\the\letterpaper
     \vsize 7.95in
  \else
     \vsize 7.65in
  \fi
  \hoffset -.75in
  \voffset -.745in
  \font\titlefont=cmbx10 \scaledmag2
  \font\headingfont=cmbx10 \scaledmag1
  \font\smallfont=cmr6
  \font\smallsy=cmsy6
  \font\eightrm=cmr8
  \font\eightbf=cmbx8
  \font\eightit=cmti8
  \font\eighttt=cmtt8
  \font\eightmi=cmmi8
  \font\eightsy=cmsy8
  \textfont0=\eightrm
  \textfont1=\eightmi
  \textfont2=\eightsy
  \def\rm{\eightrm}
  \def\bf{\eightbf}
  \def\it{\eightit}
  \def\tt{\eighttt}
  \normalbaselineskip=.8\normalbaselineskip
  \if 1\the\letterpaper
     \normalbaselineskip=.8\normalbaselineskip
  \else
     \normalbaselineskip=.7\normalbaselineskip
  \fi
  \normallineskip=.8\normallineskip
  \normallineskiplimit=.8\normallineskiplimit
  \normalbaselines\rm           %make definitions take effect

  \if 2\ncolumns
    \let\maxcolumn=b
    \footline{\hss\rm\folio\hss}
    \def\makefootline{\vskip 2in \hsize=6.86in\line{\the\footline}}
  \else \if 3\ncolumns
    \let\maxcolumn=c
    \nopagenumbers
  \else
    \errhelp{You must set \columnsperpage equal to 1, 2, or 3.}
    \errmessage{Illegal number of columns per page}
  \fi\fi

  \intercolumnskip=.46in
  \def\abc{a}
  \output={%                    %see The TeXbook page 257
      % This next line is useful when designing the layout.
      %\immediate\write16{Column \folio\abc\space starts with \firstmark}
      \if \maxcolumn\abc \multicolumnformat \global\def\abc{a}
      \else\if a\abc
        \global\setbox\columna\columnbox \global\def\abc{b}
        %% in case we never use \columnb (two-column mode)
        \global\setbox\columnb\hbox to -\intercolumnskip{}
      \else
        \global\setbox\columnb\columnbox \global\def\abc{c}\fi\fi}
  \def\multicolumnformat{\shipout\vbox{\makeheadline
      \hbox{\box\columna\hskip\intercolumnskip
        \box\columnb\hskip\intercolumnskip\columnbox}
      \makefootline}\advancepageno}
  \def\columnbox{\leftline{\pagebody}}

  \def\bye{\par\vfill\supereject
    \if a\abc \else\null\vfill\eject\fi
    \if a\abc \else\null\vfill\eject\fi
    \end}
\fi

% we won't be using math mode much, so redefine some of the characters
% we might want to talk about
\catcode`\^=12
\catcode`\_=12

\chardef\\=`\\
\chardef\{=`\{
\chardef\}=`\}

\hyphenation{mini-buf-fer}

\parindent 0pt
\parskip 1ex plus .5ex minus .5ex

\def\small{\smallfont\textfont2=\smallsy\baselineskip=.8\baselineskip}

% newcolumn - force a new column.  Use sparingly, probably only for
% the first column of a page, which should have a title anyway.
\outer\def\newcolumn{\vfill\eject}

% title - page title.  Argument is title text.
\outer\def\title#1{{\titlefont\centerline{#1}}\vskip 1ex plus .5ex}

% section - new major section.  Argument is section name.
\outer\def\section#1{\par\filbreak
  \vskip 3ex plus 2ex minus 2ex {\headingfont #1}\mark{#1}%
  \vskip 2ex plus 1ex minus 1.5ex}

\newdimen\keyindent

% beginindentedkeys...endindentedkeys - key definitions will be
% indented, but running text, typically used as headings to group
% definitions, will not.
\def\beginindentedkeys{\keyindent=1em}
\def\endindentedkeys{\keyindent=0em}
\endindentedkeys

% paralign - begin paragraph containing an alignment.
% If an \halign is entered while in vertical mode, a parskip is never
% inserted.  Using \paralign instead of \halign solves this problem.
\def\paralign{\vskip\parskip\halign}

% \<...> - surrounds a variable name in a code example
\def\<#1>{{\it #1\/}}

% kbd - argument is characters typed literally.  Like the Texinfo command.
\def\kbd#1{{\tt#1}\null}        %\null so not an abbrev even if period follows

% beginexample...endexample - surrounds literal text, such a code example.
% typeset in a typewriter font with line breaks preserved
\def\beginexample{\par\leavevmode\begingroup
  \obeylines\obeyspaces\parskip0pt\tt\tolerance=10000}
{\obeyspaces\global\let =\ }
\def\endexample{\endgroup}

% key - definition of a key.
% \key{description of key}{key-name}
% prints the description left-justified, and the key-name in a \kbd
% form near the right margin.
\def\key#1#2{\leavevmode\hbox to \hsize{\vbox
  {\hsize=.75\hsize\rightskip=1em \tolerance=20000
   \raggedright
   \hskip\keyindent\hangindent=1em\strut#1\strut}\kbd{\quad#2}\hss}}

\newbox\metaxbox
\setbox\metaxbox\hbox{\kbd{M-x }}
\newdimen\metaxwidth
\metaxwidth=\wd\metaxbox

% metax - definition of a M-x command.
% \metax{description of command}{M-x command-name}
% Tries to justify the beginning of the command name at the same place
% as \key starts the key name.  (The "M-x " sticks out to the left.)
\def\metax#1#2{\leavevmode\hbox to \hsize{\vbox
  {\hsize=.74\hsize\rightskip=1em
   \raggedright \tolerance=20000
   \hskip\keyindent\hangindent=1em\strut#1\strut\par}%
   %\hskip-\metaxwidth minus 1fil
   \kbd{#2}\hss}}

% threecol - like "key" but with two key names.
% for example, one for doing the action backward, and one for forward.
\def\threecol#1#2#3{\hskip\keyindent\relax#1\hfil&\kbd{#2}\hfil\quad
  &\kbd{#3}\hfil\quad\cr}

%**end of header


\title{Carte de r\'ef\'erence de GNU Emacs}

\centerline{(pour la version \versionemacs)}

\section{Lancement d'Emacs}

Pour lancer GNU Emacs \versionemacs, il suffit de taper son nom : \kbd{emacs}

Pour charger un fichier \`a \'editer, voir Fichiers, ci-dessous.

\section{Quitter Emacs}

\key{suspend Emacs (ou l'iconifie sous X)}{C-z}
\key{quitter d\'efinitivement Emacs}{C-x C-c}

\section{Fichiers}

\key{{\bf lire} un fichier dans Emacs}{C-x C-f}
\key{{\bf sauvegarder} un fichier sur disque}{C-x C-s}
\key{sauvegarder {\bf tous} les fichiers}{C-x s}
\key{{\bf ins\'erer} le contenu d'un autre fichier dans ce tampon}{C-x i}
\key{remplacer ce fichier par le fichier voulu}{C-x C-v}
\key{\'ecrire le tampon dans un fichier donn\'e}{C-x C-w}
\key{bascule du mode lecture-seule du tampon}{C-x C-q}

\section{Obtenir de l'aide}

Le syst\`eme d'aide est simple. Faites \kbd{C-h} (ou \kbd{F1}) et suivez
les instructions. Si vous d\'ebutez, faites \kbd{C-h t} pour suivre un
{\bf didacticiel}.

\key{supprimer la fen\^etre d'aide}{C-x 1}
\key{faire d\'efiler la fen\^etre d'aide}{C-M-v}

\key{apropos : montrer les commandes contenant une certaine cha\^\i{}ne}{C-h a}
\key{d\'ecrire la fonction lanc\'ee par une touche}{C-h k}
\key{d\'ecrire une fonction}{C-h f}
\key{obtenir des informations sp\'ecifiques au mode}{C-h m}

\section{R\'ecup\'eration des erreurs}

\key{{\bf avorter} une commande partiellement tap\'ee ou ex\'ecut\'ee}{C-g}
\metax{{\bf r\'ecup\'erer} les fichier perdus par un crash du syst\`eme}{M-x recover-session}
\metax{{\bf annuler} une modification non souhait\'ee}{C-x u, C-_ {\rm ou} C-/}
\metax{restaurer un tampon avec son contenu initial}{M-x revert-buffer}
\key{r\'eafficher un \'ecran perturb\'e}{C-l}

\section{Recherche incr\'ementale}

\key{rechercher vers l'avant}{C-s}
\key{rechercher vers l'arri\`ere}{C-r}
\key{rechercher vers l'avant par expression rationnelle}{C-M-s}
\key{rechercher vers l'arri\`ere par expression rationnelle}{C-M-r}

\key{s\'electionner la cha\^\i{}ne de recherche pr\'ec\'edente}{M-p}
\key{s\'electionner la cha\^\i{}ne de recherche suivante}{M-n}
\key{sortir de la recherche incr\'ementale}{RET}
\key{annuler l'effet du dernier caract\`ere}{DEL}
\key{annuler la recherche en cours}{C-g}

Refaites \kbd{C-s} ou \kbd{C-r} pour r\'ep\'eter la recherche dans une
direction quelconque.
Si Emacs est encore en train de chercher, \kbd{C-g} n'annule que ce
qui n'a pas \'et\'e fait.

\shortcopyrightnotice

\section{D\'eplacements}

\paralign to \hsize{#\tabskip=10pt plus 1 fil&#\tabskip=0pt&#\cr
\threecol{{\bf entit\'e sur laquelle se d\'eplacer}}{{\bf en arri\`ere}}{{\bf en avant}}
\threecol{caract\`ere}{C-b}{C-f}
\threecol{mot}{M-b}{M-f}
\threecol{ligne}{C-p}{C-n}
\threecol{aller au d\'ebut (ou \`a la fin) de la ligne}{C-a}{C-e}
\threecol{phrase}{M-a}{M-e}
\threecol{paragraphe}{M-\{}{M-\}}
\threecol{page}{C-x [}{C-x ]}
\threecol{s-expression}{C-M-b}{C-M-f}
\threecol{fonction}{C-M-a}{C-M-e}
\threecol{aller au d\'ebut (ou \`a la fin) du tampon}{M-<}{M->}
}

\key{passer \`a l'\'ecran suivant}{C-v}
\key{passer \`a l'\'ecran pr\'ec\'edent}{M-v}
\key{d\'efiler l'\'ecran vers la droite}{C-x <}
\key{d\'efiler l'\'ecran vers la gauche}{C-x >}
\key{placer la ligne courante au centre de l'\'ecran}{C-u C-l}

\section{D\'etruire et supprimer}

\paralign to \hsize{#\tabskip=10pt plus 1 fil&#\tabskip=0pt&#\cr
\threecol{{\bf entit\'e \`a supprimer}}{{\bf en arri\`ere}}{{\bf en avant}}
\threecol{caract\`ere (suppression, pas destruction)}{DEL}{C-d}
\threecol{mot}{M-DEL}{M-d}
\threecol{ligne (jusqu'\`a la fin)}{M-0 C-k}{C-k}
\threecol{phrase}{C-x DEL}{M-k}
\threecol{s-expression}{M-- C-M-k}{C-M-k}
}

\key{d\'etruire une {\bf r\'egion}}{C-w}
\key{copier une r\'egion dans le kill ring}{M-w}
\key{d\'etruire jusqu'\`a l'occurrence suivante de {\it car}}{M-z {\it car}}

\key{r\'ecup\'erer la derni\`ere chose d\'etruite}{C-y}
\key{remplacer la derni\`ere r\'ecup\'eration par ce qui a \'et\'e d\'etruit avant}{M-y}

\section{Marquer}

\key{placer la marque ici}{C-@ {\rm ou} C-SPC}
\key{\'echanger le point et la marque}{C-x C-x}

\key{placer la marque {\it arg\/} {\bf mots} plus loin}{M-@}
\key{marquer le {\bf paragraphe}}{M-h}
\key{marquer la {\bf page}}{C-x C-p}
\key{marquer la {\bf s-expression}}{C-M-@}
\key{marquer la {\bf fonction}}{C-M-h}
\key{marquer tout le {\bf tampon}}{C-x h}

\section{Remplacement interactif}

\key{remplacer interactivement une cha\^\i{}ne de texte}{M-\%}
% query-replace-regexp est liee a C-M-% mais on ne peut pas le
% taper dans une console.
\metax{en utilisant les expressions rationnelles}{M-x query-replace-regexp}

Les r\'eponses admises dans le mode de remplacement interactif sont :

\key{{\bf remplacer} celle-l\`a, passer \`a la suivante}{SPC}
\key{remplacer celle-l\`a, rester l\`a}{,}
\key{{\bf passer} \`a la suivante sans remplacer}{DEL}
\key{remplacer toutes les correspondances suivantes}{!}
\key{{\bf revenir} \`a la correspondance pr\'ec\'edente}{^}
\key{{\bf sortir} du remplacement interactif}{RET}
\key{entrer dans l'\'edition r\'ecursive (\kbd{C-M-c} pour sortir)}{C-r}

\section{Fen\^etres multiples}

Lorsqu'il y a deux commandes, la seconde est une commande identique \`a
la premi\`ere pour un cadre au lieu d'une fen\^etre.

{\setbox0=\hbox{\kbd{0}}\advance\hsize by 0\wd0
\paralign to \hsize{#\tabskip=10pt plus 1 fil&#\tabskip=0pt&#\cr
\threecol{supprimer toutes les autres fen\^etres}{C-x 1\ \ \ \ }{C-x 5 1}
\threecol{diviser la fen\^etre horizontalement}{C-x 2\ \ \ \ }{C-x 5 2}
\threecol{supprimer cette fen\^etre}{C-x 0\ \ \ \ }{C-x 5 0}
}}
\key{diviser la fen\^etre verticalement}{C-x 3}

\key{faire d\'efiler l'autre fen\^etre}{C-M-v}

{\setbox0=\hbox{\kbd{0}}\advance\hsize by 2\wd0
\paralign to \hsize{#\tabskip=10pt plus 1 fil&#\tabskip=0pt&#\cr
\threecol{placer le curseur dans une autre fen\^etre}{C-x o}{C-x 5 o}

\threecol{s\'electionner le tampon dans l'autre fen\^etre}{C-x 4 b}{C-x 5 b}
\threecol{afficher le tampon dans l'autre fen\^etre}{C-x 4 C-o}{C-x 5 C-o}
\threecol{charger un fichier dans l'autre fen\^etre}{C-x 4 f}{C-x 5 f}
\threecol{charger un fichier en lecture seule dans l'autre fen\^etre}{C-x 4 r}{C-x 5 r}
\threecol{lancer Dired  dans l'autre fen\^etre}{C-x 4 d}{C-x 5 d}
\threecol{trouver un tag dans l'autre fen\^etre}{C-x 4 .}{C-x 5 .}
}}

\key{agrandir la fen\^etre}{C-x ^}
\key{rapetisser la fen\^etre}{C-x \{}
\key{\'elargir la fen\^etre}{C-x \}}

\section{Formater}

\key{indenter la {\bf ligne} courante (d\'epend du mode)}{TAB}
\key{indenter la {\bf r\'egion} courante (d\'epend du mode)}{C-M-\\}
\key{indenter la {\bf s-expression} courante (d\'epend du mode)}{C-M-q}
\key{indenter la r\'egion sur {\it arg\/} colonnes}{C-x TAB}

\key{ins\'erer un newline apr\`es le point}{C-o}
\key{d\'eplacer le reste de la ligne vers le bas}{C-M-o}
\key{supprimer les lignes blanches autour du point}{C-x C-o}
\key{joindre la ligne \`a la pr\'ec\'edente (\`a la suivante avec arg)}{M-^}
\key{supprimer tous les espaces autour du point}{M-\\}
\key{mettre exactement un espace \`a l'emplacement du point}{M-SPC}

\key{remplir le paragraphe}{M-q}
\key{placer la marge droite}{C-x f}
\key{d\'efinir le pr\'efixe par lequel commencera chaque ligne}{C-x .}

\key{d\'efinir la fonte}{M-o}

\section{Modifier la casse}

\key{mettre le mot en majuscules}{M-u}
\key{mettre le mot en minuscules}{M-l}
\key{mettre le mot en capitales}{M-c}

\key{mettre la r\'egion en majuscules}{C-x C-u}
\key{mettre la r\'egion en minuscules}{C-x C-l}

\section{Le mini-tampon}

Les touches suivantes sont utilisables dans le mini-tampon :

\key{compl\'eter autant que possible}{TAB}
\key{compl\'eter un mot}{SPC}
\key{compl\'eter et ex\'ecuter}{RET}
\key{montrer les compl\'etions possibles}{?}
\key{rechercher l'entr\'ee pr\'ec\'edente du mini-tampon}{M-p}
\key{rechercher l'entr\'ee suivante du mini-tampon ou le d\'efaut}{M-n}
\key{rechercher \`a rebours par expr. rationnelle dans l'historique}{M-r}
\key{rechercher vers l'avant par expr. rationnelle dans l'historique}{M-s}
\key{annuler la commande}{C-g}

Faites \kbd{C-x ESC ESC} pour \'editer et r\'ep\'eter la derni\`ere commande
ayant utilis\'e le minitampon. Faites \kbd{F10} pour activer la barre de
menu utilisant le minitampon.

\newcolumn
\title{Carte de r\'ef\'erence de GNU Emacs}

\section{Tampons}

\key{s\'electionner un autre tampon}{C-x b}
\key{\'enum\'erer tous les tampons}{C-x C-b}
\key{supprimer un tampon}{C-x k}

\section{Transposer}

\key{transposer des {\bf caract\`eres}}{C-t}
\key{transposer des {\bf mots}}{M-t}
\key{transposer des {\bf lignes}}{C-x C-t}
\key{transposer des {\bf s-expressions}}{C-M-t}

\section{V\'erifier l'orthographe}

\key{v\'erifier l'orthographe du mot courant}{M-\$}
\metax{v\'erifier l'orthographe de tous les mots d'une r\'egion}{M-x ispell-region}
\metax{v\'erifier l'orthographe de tout le tampon}{M-x ispell-buffer}

\section{Tags}

\key{trouver un tag (une d\'efinition)}{M-.}
\key{trouver l'occurrence suivante du tag}{C-u M-.}
\metax{sp\'ecifier un nouveau fichier de tags}{M-x visit-tags-table}

\metax{rechercher par expr. rationnelles dans tous les fichiers du
  tableau de tags}{M-x tags-search}
\metax{lancer un remplacement interactif sur tous les fichiers}{M-x tags-query-replace}
\key{continuer la derni\`ere recherche de tags ou le remplacement interactif}{M-,}

\section{Shells}

\key{ex\'ecuter une commande shell}{M-!}
\key{lancer une commande shell sur la r\'egion}{M-|}
\key{filtrer la r\'egion avec une commande shell}{C-u M-|}
\key{lancer un shell dans la fen\^etre \kbd{*shell*}}{M-x shell}

\section{Rectangles}

\key{copier le rectangle dans le registre}{C-x r r}
\key{d\'etruire le rectangle}{C-x r k}
\key{r\'ecup\'erer le rectangle}{C-x r y}
\key{ouvrir le rectangle, en d\'ecalant le texte \`a droite}{C-x r o}
\key{vider le rectangle}{C-x r c}
\key{pr\'efixer chaque ligne avec une cha\^\i{}ne}{C-x r t}

\section{Abr\'eviations}

\key{ajouter une abr\'eviation globale}{C-x a g}
\key{ajouter une abr\'eviation locale au mode}{C-x a l}
\key{ajouter une expansion globale pour cette abr\'eviation}{C-x a i g}
\key{ajouter une expansion locale au mode pour cette abr\'eviation}{C-x a i l}
\key{faire une expansion explicite de cette abr\'eviation}{C-x a e}

\key{faire une expansion dynamique du mot pr\'ec\'edent}{M-/}

\section{Expressions rationnelles}

\key{un unique caract\`ere quelconque, sauf une fin de ligne}{. {\rm(point)}}
\key{z\'ero r\'ep\'etition ou plus}{*}
\key{une r\'ep\'etition ou plus}{+}
\key{z\'ero ou une r\'ep\'etition}{?}
\key{caract\`ere sp\'ecial pour quoter l'expression rationnelle {\it c\/}}{\\{\it c}}
\key{alternative (``ou'')}{\\|}
\key{regroupement}{\\( {\rm$\ldots$} \\)}
\key{le m\^eme texte que dans le {\it n\/}i\`eme groupe}{\\{\it n}}
\key{limite de mot}{\\b}
\key{non limite de mot}{\\B}

\paralign to \hsize{#\tabskip=10pt plus 1 fil&#\tabskip=0pt&#\cr
\threecol{{\bf entit\'e}}{{\bf d\'ebut}}{{\bf fin}}
\threecol{ligne}{^}{\$}
\threecol{mot}{\\<}{\\>}
\threecol{tampon}{\\`}{\\'}

\threecol{{\bf classe de caract\`ere}}{{\bf correspond}}{{\bf correspond
    \`a d'autres}}
\threecol{ensemble explicite}{[ {\rm$\ldots$} ]}{[^ {\rm$\ldots$} ]}
\threecol{caract\`ere de mot}{\\w}{\\W}
\threecol{caract\`ere avec la syntaxe {\it c}}{\\s{\it c}}{\\S{\it c}}
}

\section{Jeux de caract\`eres internationaux}

\key{indiquer la langue principale}{C-x RET l}
\metax{montrer toutes les m\'ethodes de saisie}{M-x list-input-methods}
\key{activer ou d\'esactiver la m\'ethode de saisie}{C-\\}
\key{choisir le syst\`eme de codage pour la commande suivante}{C-x RET c}
\metax{montrer tous les syst\`emes de codage}{M-x list-coding-systems}
\metax{choisir le syst\`eme de codage pr\'ef\'er\'e}{M-x prefer-coding-system}

\section{Info}

\key{entrer dans le visualisateur de la documentation Info}{C-h i}
\key{chercher une fonction ou une variable pr\'ecise dans Info}{C-h S}
\beginindentedkeys

Se d\'eplacer dans un n\oe{}ud :

\key{une page plus bas}{SPC}
\key{une page plus haut}{DEL}
\key{d\'ebut du n\oe{}ud}{. {\rm (point)}}

Passer de n\oe{}ud en n\oe{}ud :

\key{n\oe{}ud {\bf suivant}}{n}
\key{n\oe{}ud {\bf pr\'ec\'edent}}{p}
\key{aller {\bf plus haut}}{u}
\key{choisir un sujet de menu par son nom}{m}
\key{choisir le {\it n\/}i\`eme sujet de menu par son num\'ero (1--9)}{{\it n}}
\key{suivre une r\'ef\'erence crois\'ee  (on revient avec \kbd{l})}{f}
\key{revenir au dernier n\oe{}ud visit\'e}{l}
\key{revenir au n\oe{}ud du r\'epertoire}{d}
\key{aller au n\oe{}ud de plus haut niveau du fichier Info}{t}
\key{aller sur n'importe quel n\oe{}ud par son nom}{g}

Autres :

\key{lancer le {\bf didacticiel} Info}{h}
\key{chercher un sujet dans l'index}{i}
\key{rechercher les n\oe{}uds avec une expression rationnelle}{s}
\key{{\bf quitter} Info}{q}

\endindentedkeys

\section{Registres}

\key{sauver la r\'egion dans un registre}{C-x r s}
\key{ins\'erer le contenu du registre dans le tampon}{C-x r i}

\key{sauver la valeur du point dans un registre}{C-x r SPC}
\key{sauter au point sauv\'e dans le registre}{C-x r j}

\section{Macros clavier}

\key{{\bf lancer} la d\'efinition d'une macro clavier}{C-x (}
\key{{\bf terminer} la d\'efinition d'une macro clavier}{C-x )}
\key{{\bf ex\'ecuter} la derni\`ere macro clavier d\'efinie}{C-x e}
\key{ajouter \`a la derni\`ere macro clavier}{C-u C-x (}
\metax{donner un nom \`a la derni\`ere macro clavier}{M-x name-last-kbd-macro}
\metax{ins\'erer une d\'efinition Lisp dans le tampon}{M-x insert-kbd-macro}

\section{Commandes de gestion d'Emacs Lisp}

\key{\'evaluer la {\bf s-expression} situ\'ee avant le point}{C-x C-e}
\key{\'evaluer la {\bf defun} courante}{C-M-x}
\metax{\'evaluer la {\bf r\'egion}}{M-x eval-region}
\key{lire et \'evaluer le mini-tampon}{M-:}
\metax{charger \`a partir du r\'epertoire syst\`eme standard}{M-x load-library}

\section{Personnalisation simple}

\metax{personnaliser les variables et les fontes}{M-x customize}

% The intended audience here is the person who wants to make simple
% customizations and knows Lisp syntax.

Cr\'eation de liaisons de touches globales en Emacs Lisp (exemples):

\beginexample%
(global-set-key "\\C-cg" 'goto-line)
(global-set-key "\\M-\#" 'query-replace-regexp)
\endexample

\section{\'Ecriture de commandes}

\beginexample%
(defun \<nom-commande> (\<args>)
  "\<documentation>" (interactive "\<template>")
  \<body>)
\endexample

Exemple :

\beginexample%
(defun cette-ligne-en-haut-de-la-fenetre (line)
  "Repositionne la ligne du point en haut de la fenetre.
Avec ARG, place le point sur la ligne ARG."
  (interactive "P")
  (recenter (if (null line)
                0
              (prefix-numeric-value line))))
\endexample

La sp\'ecification \kbd{interactive} indique comment lire
interactivement les param\`etres. Faites \kbd{C-h f interactive} pour
plus de pr\'ecisions.

\copyrightnotice

\bye

% Local variables:
% compile-command: "tex fr-refcard"
% End:

% arch-tag: 39d6dc6e-1a4a-4071-84db-4719d4e9e40d
